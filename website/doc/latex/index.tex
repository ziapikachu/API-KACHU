Welcome to \hyperlink{namespace_a_p_i_kachu}{A\+P\+I\+Kachu}\textquotesingle{}s documentation. You can find a list of classes and their methods in the toolbar above.

To create a module, you have to implement the I\+Module interface and its methods accordingly. You\textquotesingle{}ll find more details in the documentation of the class.

First of all, when the server receives a new connection, it should create an instance of I\+Http\+Client containing the IP address of the client as an I\+I\+P\+Address. This is to be used later on by the modules if needed. Note that all enabled modules with the M\+O\+D\+U\+L\+E\+\_\+\+T\+Y\+P\+E\+\_\+\+C\+O\+N\+N\+E\+C\+T\+I\+ON type set will be called at that time. You can for example set up an S\+SL connection here within a module.

When the server receives a request from a known client, it has to create an I\+Http\+Transaction containing the client sending it, and set its Byte\+Buffer to the content of the received packet. This is when modules with M\+O\+D\+U\+L\+E\+\_\+\+T\+Y\+P\+E\+\_\+\+O\+N\+\_\+\+R\+E\+AD will be executed. At this point, the server should create the I\+Http\+Request, set it in the I\+Http\+Transaction object, and pass this to modules with M\+O\+D\+U\+L\+E\+\_\+\+T\+Y\+P\+E\+\_\+\+I\+N\+IT. The next step is parsing the headers ; this should be done by one of the modules with M\+O\+D\+U\+L\+E\+\_\+\+T\+Y\+P\+E\+\_\+\+P\+A\+R\+S\+I\+NG set. After this, the server has to create the I\+Http\+Response, and set it in the I\+Http\+Transaction object \+: this is when the content of the response will get filled, with the M\+O\+D\+U\+L\+E\+\_\+\+T\+Y\+P\+E\+\_\+\+P\+R\+O\+C\+E\+SS modules. You have to create a module retrieving the content of the file there and fill the Byte\+Buffer within the I\+Http\+Response. Modules with their type set to M\+O\+D\+U\+L\+E\+\_\+\+T\+Y\+P\+E\+\_\+\+P\+O\+S\+T\+\_\+\+P\+R\+O\+C\+E\+SS will get executed right after. The response is almost ready to be sent by then ; modules with M\+O\+D\+U\+L\+E\+\_\+\+T\+Y\+P\+E\+\_\+\+O\+N\+\_\+\+W\+R\+I\+TE will be executed, and then the server can send the response. Lastly, after the response is sent, modules with M\+O\+D\+U\+L\+E\+\_\+\+T\+Y\+P\+E\+\_\+\+E\+ND sent will be executed on the transaction, potentially doing last actions.

The execution routine of the modules returns a boolean, whose behaviour changes with the module\textquotesingle{}s e\+Module\+Flags. A flag of M\+O\+D\+U\+L\+E\+\_\+\+F\+L\+A\+G\+S\+\_\+\+N\+O\+NE specifies that the return value should be ignored, and have no impact on the execution of other modules. The M\+O\+D\+U\+L\+E\+\_\+\+F\+L\+A\+G\+S\+\_\+\+R\+E\+Q\+U\+I\+R\+ED flag specifies that if the return value is false, the execution of modules should be stopped and the response sent directly (the module has to set the appropriate error code). If the M\+O\+D\+U\+L\+E\+\_\+\+F\+L\+A\+G\+S\+\_\+\+S\+U\+F\+F\+I\+C\+I\+E\+NT is set, a return value of true specifies that only other modules with M\+O\+D\+U\+L\+E\+\_\+\+F\+L\+A\+G\+\_\+\+R\+E\+Q\+U\+I\+R\+ED should be executed before sending the response. Last, if both of those are set, a return value of true means that the execution of other modules should be stopped and the request sent directly. Please note that in each of those cases, the modules of type M\+O\+D\+U\+L\+E\+\_\+\+T\+Y\+P\+E\+\_\+\+O\+N\+\_\+\+W\+R\+I\+TE and M\+O\+D\+U\+L\+E\+\_\+\+T\+Y\+P\+E\+\_\+\+E\+ND should still be executed. 